\section{Introduction}

{
	\color{red} \bfseries
	State the objectives of the assignment. Summarise briefly your preparation work,  your experimental work,, and results achieved. Specifically, state which parts of the assignment were delivered according to the requirements and summarise any extensions to the basic specification you have carried out with references to the sections.  ( approx. 0.5 page).
}

\section{Instruction format, decoder design, program memory, and program counter}

{
	\color{red} \bfseries
	Provide a block diagram of your picoMIPS design showing the sizes of all the busses and modules.
	
	Describe your picoMIPS instruction format and the instructions you have implemented in your decoder. Give a listing of your program implemented in the Program Memory. You can show snippets of your source code. There is no need to show the full source code for all your modules in the report as the full source code must be submitted separately.  Do not copy any code or diagrams from the lectures and picoMIPS SystemVerilog files provided on the ELEC6016 notes site. Give your Modelsim testbenches and Modelsim results.  DO NOT make statements such as: “Figure 2 shows the simulation results of the module functioning correctly”.  Instead, explain the results shown in the figures to demonstrate that you understand how the tested modules work.  You can show RTL level diagrams from Quartus if you wish. (max 2.5 pages).
}

\section{General Purpose Register file design, simulation and synthesis}

{
	\color{red} \bfseries
	As above (max 1.5 page)
}

\section{Arithmetic Logic Unit and Mulitiplier Design}

{
	\color{red} \bfseries
	 Explain the functions implemented in  your ALU and explain your testbench. Show Modelsim test results.  If you have implemented a hardware multiplier (or multipliers), explain your multiplier design and give Modelsim test results. State if your multiplier module synthesised as an embedded hardware multiplier.  (approx. 1.5- 2 pages)
}

\section{Altera DE0 implementation}

{
	\color{red} \bfseries
	Explain how you tested your design after  programming the FPGA. In case you had to edit your original code and resynthesize – explain what you did.  ( approx. 1-2 pages)
}

\section{Conclusion}

{
	\color{red} \bfseries
	State which objectives listed in your Introducton have been achieved. Calculate the cost figure of your design for synthesis on a Cyclone IV E.. Give your general  conclusion, comment on what you learnt.  Comment on ways to improve the design or extend it further.  ( approx.0.25 – 0.5 of a page)
}