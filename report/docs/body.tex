\section{Introduction} \label{sec:intro}

This project relates to the design and implementation of a picoMIPS processor. The objectives are:
\begin{enumerate}
	\item Design of a picoMIPS processor, capable of executing an affine transform. \label{obj:design}
	\item Implementation of the designed processor in SystemVerilog. \label{obj:imp}
	\item Verification of the SystemVerilog model, by simulation. \label{obj:verif}
	\item Validation of the synthesised SystemVerilog processor. \label{obj:valid}
	\item Minimising the resources used by the synthesised processor design. \label{obj:minimal}
\end{enumerate}

The affine transform which the processor is to execute can be represented by Equation \ref{eq:affine-transform}. A assembly program was written to implement this, and the program was then translated to machine code using a custom assembler written for the purpose.

\begin{equation}
\begin{bmatrix}
x_2\\
y_2
\end{bmatrix}
=
\begin{bmatrix}
A_{11} & A_{12}\\
A_{21} & A_{22}
\end{bmatrix}
\begin{bmatrix}
x_1\\
y_1
\end{bmatrix}
+
\begin{bmatrix}
B_1\\
B_2
\end{bmatrix}
\label{eq:affine-transform}
\end{equation}

The minimalisation objective, objective \ref{obj:minimal} is quanified by by the cost figure of Equation \ref{eq:cost}.

\begin{equation}
	\text{Cost} = [\text{No. of Logic Elements}] + \max([\text{No. of 9-bit Multipliers used}] -2, 0) + \frac{ [\text{kBits of RAM}]}{1024} \times 30
	\label{eq:cost}
\end{equation}

In preparation for the assignment, I conducted research into minimal instruction set computers, and designed my system on paper. The design of the instruction set is summarised in Section \ref{sec:inst-format}, and the design of the remainder of the system is discussed therein, as well as in subsequent sections.

As part of this preparation I investigated the structure of Altera logic blocks, and synthesised small test modules in order to see their resource utilisation.

As an extension exercise I wrote SystemVerilog code to convert the signed 8-bit words on the switches and LEDs to binary coded decimal and then display this using the seven segment displays present on the development board. This is discussed briefly in Section \ref{sec:implementation}.

The simulation timing diagrams reproduced in the report have been exported from ModelSim, and converted to a Ti\textit{k}Z timing waveform using modelsim2latex \cite{show2016}. Trivial modifications were made to make the script compatible with the files exported from ModelSim. In each case, the waveform rendered by Ti\textit{k}Z has been manually validated against the displayed result in ModelSim, to ensure correct operation of the tool.


\section{System Design and Verification} \label{sec:design-verif}

\subsection{Overview} \label{sec:sys-overview}

A block diagram showing my picoMIPS implementation is shown in Figure \ref{fig:final-proc-arch}. In this diagram, blue lines represent address signals, red represents data signals, and green represents control signals. Each block in this diagram represents a SystemVerilog module in the design, and each module has been annotated with the bus width of all of the signals into and out of it.

The cycle counter defines the current stage of instruction execution. The processor is not pipelined, so the cycle counter controls which parts of the processor are active on any given clock cycle. The counter has three states, decode, execute, and write, which make up one instruction cycle.

The program counter stores the instructions which make up the program, and on each instruction cycle it presents the instruction pointed to by it's address input to the processor.

The register block contains a small amount of storage necessary for program execution. Two registers are read per instruction cycle, and the result of the \gls{alu} operation is written back to the second register.

The \gls{alu} performs the requested operations on the input registers, and presents the result to the rest of the processor so that the appropriate action can be carried out.

The final block is the program address multiplexer. This block is responsible for presenting the program memory with either the next address, or a branch address depending on the \gls{alu} result. In a traditional processor, the next address would be calculated and stored by a program counter. This processor uses a slightly different approach and stores the next address is stored in the program code instead. The reason for this choice is that it contributes less to the cost function. The processor uses 5 bit addresses, and therefore a 5 bit counter would be needed. This would require 5 logic elements, which is a cost of 5. There are 31 instructions in the main program, and so the cost of adding 5 bits to each one is $\frac{31 \times 5}{1024}\times 30 = 4.54$, therefore it is cheaper to not have a dedicated program counter.

\begin{figure}[ht]
	\centering
	\def \nodeDist{2.5cm}
	
	\tikzstyle{cycleBlock} = [rectangle, rounded corners, minimum width=\nodeDist, minimum height=\nodeDist, text centred, draw=black, fill=black!40]
	\tikzstyle{normalBlock} = [cycleBlock] %Didn't end up using this in the end
	\tikzstyle{arrowNml} = [thick,->,>=stealth]
	\tikzstyle{arrowRev} = [thick,<-,>=stealth]
	\tikzstyle{arrowDbl} = [thick,<->,>=stealth]
	\tikzstyle{lineNml} = [thick]
	\begin{tikzpicture}[node distance=\nodeDist, scale=0.7, every node/.style={transform shape}]
	
	\newcommand{\buswidth}[5]{\draw ([xshift=#4,yshift=#5] #1) node [#3=.6ex,] {#2} +(45:-.8ex) -- +(45:.8ex)};
	

	% program memory
	\node (progMem) [normalBlock, anchor=south west, minimum height=9cm] at (0,0) {\shortstack{Program\\Memory}};
	
	\draw [arrowRev, draw=green] ( [yshift=-1cm] progMem.west) to node[above] {Cycle} ++(-0.5*\nodeDist,0);
	\buswidth{progMem.west}{$2$}{below}{-0.5*\nodeDist +0.25cm}{-1cm};
	
	\draw [arrowRev, draw=blue] ( progMem.west) to node[above] {Address} ++(-\nodeDist,0) to ++(0,-0.25*\nodeDist - 2cm - 4.5cm) to ++(\nodeDist,0) coordinate(addrLine);
	\buswidth{addrLine}{$5$}{below}{-0.25cm}{0cm};
	
	\draw [arrowNml, draw=blue] ( [yshift=-2cm] progMem.north east) to node[above] {  \shortstack{Register\\Address 1}} ++(\nodeDist,0);
	\buswidth{progMem.north east}{$3$}{below}{0.25cm}{-2cm};
	\draw [arrowNml, draw=blue] ( [yshift=-3cm] progMem.north east) to node[above] {  \shortstack{Register\\Address 2}} ++(\nodeDist,0);
	\buswidth{progMem.north east}{$3$}{below}{0.25cm}{-3cm};
	
	\draw [arrowNml, draw=green] ( [yshift=3cm] progMem.south east) to node[above] {  \shortstack{Opcode}} ++(3*\nodeDist,0);
	\buswidth{progMem.south east}{$1$}{below}{0.25cm}{3cm};
	
	
	% registers
	\node (regs) [cycleBlock, minimum height=5cm, right=of progMem.north east, anchor=north west] {\shortstack{Registers}};
	\draw [arrowNml, draw=red] ( [yshift=-1cm] regs.north east) to node[above] {  \shortstack{Register 1\\Data}} ++(\nodeDist,0);
	\buswidth{regs.north east}{$8$}{below}{0.25cm}{-1cm};
	\draw [arrowNml, draw=red] ( [yshift=-2cm] regs.north east) to node[above] {  \shortstack{Register 2\\Data}} ++(\nodeDist,0);
	\buswidth{regs.north east}{$8$}{below}{0.25cm}{-2cm};
	
	\draw [arrowRev, draw=red] ( [yshift=-1cm] regs.north west) to node[above] {  \shortstack{Write\\Data}} ++(-0.5*\nodeDist,0)  to ++(0,0.5*\nodeDist) to ++(4*\nodeDist,0) to ++(0,-0.5*\nodeDist) to ++(-0.5*\nodeDist,0) coordinate(writeDataEnd);
	\buswidth{writeDataEnd}{$8$}{below}{0.25cm}{0cm};
	\draw [arrowRev, draw=green] ( [yshift=1cm] regs.south west) to node[above] {Cycle} ++(-0.5*\nodeDist,0);
	\buswidth{regs.south west}{$2$}{below}{-0.5*\nodeDist +0.25cm}{1cm};
	
	% alu
	\node (alu) [normalBlock, minimum height=8cm, right=of regs.north east, anchor=north west] {\shortstack{ALU}};
	\draw [arrowRev, draw=red] ( [yshift=1cm] alu.south west) to node[above] {  \shortstack{Immediate}} ++(-1.5*\nodeDist,0) coordinate(aluLineHalfway);
	\draw[lineNml] (aluLineHalfway) -- ++(-1.5*\nodeDist,0) coordinate(aluLineStart);
	\buswidth{aluLineStart}{$5$}{below}{0.25cm}{0cm};
	
	% Program counter
	\node (pc) [cycleBlock, minimum height=4cm, below= 0.25*\nodeDist of progMem.south, anchor=north] {\shortstack{Program\\Address\\Multiplexer}};
	\draw [arrowRev, draw=green] ( [yshift=1cm] pc.south east) to node[above] {Branch} ++(4.5*\nodeDist,0) to ++(0,0.25*\nodeDist +5cm) to ++(-0.5*\nodeDist,0) coordinate(branchEnd);
	\buswidth{branchEnd}{$1$}{below}{0.25cm}{0cm};
	\draw [arrowRev, draw=blue] ( [yshift=2cm] pc.south east) to node[above] {Branch Address} ++(1.5*\nodeDist,0) to ++(0,0.25*\nodeDist +4cm);
	
	\draw [arrowRev, draw=blue] ( [yshift=3cm] pc.south east) to node[above] {Next Address} ++(1*\nodeDist,0) to ++(0,0.25*\nodeDist +2cm) to ++(-1*\nodeDist,0) coordinate(nextAddressEnd);
	\buswidth{nextAddressEnd}{$5$}{below}{0.25cm}{0cm};
	
	% Cycle counter
	\node (cycle) [cycleBlock, minimum height=1.5cm, above= 0.25*\nodeDist of progMem.north, anchor=south] {\shortstack{Cycle\\Counter}};
	\draw [arrowNml, draw=green] (cycle.east) to node[above] {  \shortstack{Cycle}} ++(0.5*\nodeDist,0);
	\buswidth{cycle.east}{$2$}{below}{0.25cm}{0cm};
	
	\end{tikzpicture}
	\caption{Final processor architecture}
	\label{fig:final-proc-arch}
\end{figure}

\subsection{Cycle Counter Design} \label{sec:cycle-design}

As mentioned in Section \ref{sec:sys-overview}, the counter has three states. This is represented using a one hot encoding, with the zero state also valid. This arrangement means that the counter uses the same number of bits than the equivalent binary counter. Careful assignment of state encodings means that no decoder is necessary for decoding the current cycle anywhere in the design. This is possible because no signals need to be asserted in the decode state, and so this state can be encoded as zero.

A timing diagram showing the execution of the processor across these cycles is shown in Figure \ref{fig:proc-timing-diagram}. During the decode stage, the new instruction is valid, and so it can be decoded, and the register values can be fetched from \gls{ram}. During the execute stage, the register values have been fetched from \gls{ram} and so the \gls{alu} can perform the operation. During the write cycle the data produced by the \gls{alu} is written back to \gls{ram}. A cycle is necessary for this because the \gls{ram} blocks inside the Cyclone IV \gls{fpga} do not support read during write with new data, so a cycle is necessary to ensure that the register value is valid for the decode cycle.

\begin{figure}[ht]
	\centering
	\begin{tikztimingtable}%[ xscale=0.8, yscale=1.5]
		Clock			& L 12{2C} C \\
		Cycle			& D{} 2{ 4D{Decode} 4D{Execute} 4D{Write}} D{}\\
		PC				& D{} 2{4X 8D{Valid}}  X \\
		Instruction		& D{} 12D{Instruction[n]} 12D{Instruction[n+1]} D{} \\
		Register 1 Data	& X 2{ 4X 8D{Valid}} X \\
		Register 2 Data	& X 2{ 4X 8D{Valid}} X \\
		Write Enable	& L 2{4L 4H  4L} L \\
		Branch			& L 2{4L 8D{Valid} } L \\
		\extracode
		\begin{pgfonlayer}{background}
			\begin{scope}[semitransparent ,semithick]
				\vertlines[darkgray,dotted]{1.0,5.0,...,27.0}
			\end{scope}
		\end{pgfonlayer}
	\end{tikztimingtable}
	\caption{Processor timing diagram}
	\label{fig:proc-timing-diagram}
\end{figure}

\subsection{Instruction and Program Design} \label{sec:inst-format}
The processor is capable of performing two instructions tabulated in Table \ref{tbl:insts}. \texttt{SUBLEQ} was chosen as the primary instruction because it is Turing complete by itself, so can execute any program. Using solely \texttt{SUBLEQ} instructions does have a downside however. Executing a multiply operation would require many lines of code, and so would take a large amount of time, as well as a lot of program memory. For this reason a second instruction \texttt{MULTI} is also used. This instruction also gives a method of loading immediates (by multiplying by $1$), which would require an additional multiplexer and bit in the instruction in a \texttt{SUBLEQ} only system.

\begin{table}[ht]

	\centering

	\caption{Instructions}
	\label{tbl:insts}
	\begin{tabular}{l  l l}
		\toprule
		Mnemonic & Mathematical operation & English Description \\
		\midrule
		\texttt{SUBLEQ} & $\text{regs}[b] = \text{regs}[b] - \text{regs}[a];$ & \multirow{2}{0.35\textwidth}{Subtract and branch if less than, or equal to, zero.} \\
		& $if(\text{regs}[b] \leq 0) \text{ branch};$ & \\
		\midrule
		\texttt{MULTI} & $\text{regs}[b] = \text{regs}[a] \times \text{Immediate};$ & Multiply immediate.  \\
		\bottomrule
	\end{tabular}
\end{table}

The processor uses 17 bit instructions. This breaks down into a one bit opcode, two 3 bit register addresses, a 5 bit immediate / branch address, and a 5 bit value to store the next address, as shown in Figure \ref{fig:inst-format}.

The opcode can be made one bit long because there are only two instructions in the processor, and therefore one bit is enough to differentiate them. The immediate / branch address bits can be shared across the two functions because the MULTI instruction uses an immediate but does not branch, whilst the SUBLEQ instruction does not use an immediate, but can branch. The details of why these two instructions were chosen, and the details of their implementation is given in Section \ref{sec:alu-design}.


\begin{figure}[ht]
	\centering
	\begin{tabularx}{\textwidth}{Y *{16}{|Y}}
		16 & 15 & 14 & 13 & 12 & 11 & 10 & 9 & 8 & 7 & 6 & 5 & 4 & 3 & 2 & 1 & 0 \\
		\hline
		 {op.} & \multicolumn{3}{c|}{Reg. 1 Addr.} & \multicolumn{3}{c|}{Reg. 2 Addr.} & \multicolumn{5}{c|}{Imm. / Branch Address} & \multicolumn{5}{c}{Next Address} \\
		
	\end{tabularx}
	\caption{Instruction Format and Assembler}
	\label{fig:inst-format}
\end{figure}

A listing of the program used is given in Listing \ref{lst:main-prog}. This is a custom dialect of assembly code created for this project. Upon reading the source code, the reader will notice that the majority of assembly instructions used are not present as machine instructions in the processor architecture. This is because writing assembly code solely using \texttt{SUBLEQ} and \texttt{MULTI} instructions can be very confusing. For this reason a two-stage compilation toolchain was created using the Python programming language. The first stage takes each assembly instruction that doesn't map into a machine instruction, and re-writes it so that only \texttt{SUBLEQ} and \texttt{MULTI} instructions are used. This is performed by \texttt{optimiser.py}, and the second stage is to compile this assembly code into machine instructions. This is performed by \texttt{assembler.py}. The machine code output by the compilation toolchain is listed in Listing \ref{lst:main-prog-compiled}.

The majority of instructions such as \texttt{MOV} and \texttt{ADD} have their conventional definitions, however there are some slightly more esoteric instructions, namely \texttt{JLEZ}, and \texttt{JGZ}. These are represent `jump if less than or equal to zero', and `jump if greater than or equal to zero' respectively. These instructions are used to poll switch 8. Initially the more conventional \texttt{JZ}, and \texttt{JNZ} (`jump if zero', and `jump if not zero') were used. The system worked with these instructions, however they require more \texttt{SUBLEQ} instructions to implement, and since the switch 8 register is guaranteed to be either $0x00$, or $0x01$ the simpler instructions are functionally equivalent.

\subsection{Program Memory Design}
The program memory block is very simple, it consists solely of a block of synchronous \gls{rom}, initialised with the data from Listing \ref{lst:main-prog-compiled}.

Initially the \gls{rom} was inferred by creating an array of words, and then transferring the addressed word to the output on each rising clock edge, using a non blocking assignment within a rising clock edge \texttt{always} block. However later the \gls{rom} was instantiated using a dedicated Altera RAM library element. The reason for this choice is that the design requires access to the asynchronous clear input on the address register in order to reset the design. This is something which was not possible to infer using standard SystemVerilog code.

Due to its simplicity this block was not tested with an individual testbench, and the functionality of the block was verified during system level testing, as described in Section \ref{sec:sys-level-testing}.

\subsection{Program Address Multiplexer Design} \label{sec:mux-design}
The multiplexer used to multiplex between the next address and branch address is not implemented using a traditional structure, instead a multiplier is used.

The reason for this choice is that an $n$ bit multiplexer requires $n$ logic blocks for a traditional implementation, however using hardware multipliers, one can implement up to a $9$ bit multiplexer using a single $18 \times 18$ hardware multiplexer. This would represent a cost of $2$ (because an $18 \times 18$ multiplexer is formed of two $9\times9$ multipliers).

The operation of the multiplexer is illustrated by Figure \ref{fig:mux-operation}. If the two words to be multiplexed $a$ and $b$ are concatenated to a single multiplier input, and a single bit is set in the second word, then the input can be shifted by a set amount. Setting the second input to either $1$ or $1 << (\text{Word Width})$, therefore acts as a multiplexer if we observe the output bits from $[(\text{Word Width}) \times 2: (\text{Word width}) + 1]$ we have effectively created a multiplexer.

\begin{figure}[ht]
	\centering
	\begin{tabular}{r *{8}{|c}}
		Input & $b[3]$ & $b[2]$ & $b[1]$ & $b[0]$ & $a[3]$ & $a[2]$ & $a[1]$ & $a[0]$ \\
		Sel & $0$ & $0$ & $0$ & $0$ & $0$ & $0$ & $0$ & $1$  \\
		Result & \shadeCell $b[3]$ & \shadeCell $b[2]$ & \shadeCell $b[1]$ & \shadeCell $b[0]$ & $a[3]$ & $a[2]$ & $a[1]$ & $a[0]$ \\
		\midrule
		Input & $b[3]$ & $b[2]$ & $b[1]$ & $b[0]$ & $a[3]$ & $a[2]$ & $a[1]$ & $a[0]$ \\
		Sel & $0$ & $0$ & $0$ & $1$ & $0$ & $0$ & $0$ & $0$  \\
		Result & \shadeCell $a[3]$ & \shadeCell $a[2]$ & \shadeCell $a[1]$ & \shadeCell $a[0]$ & $0$ & $0$  & $0$  & $0$  \\
	\end{tabular}
	\caption{4-bit Multiplexer Operation}
	\label{fig:mux-operation}
\end{figure}

This multiplexer was designed as a parametric model, so that it can be used as the program address multiplexer, as well as elsewhere in the design.


\subsubsection{Testing}
In order to test the multiplexer. I wrote a testbench which instantiates an $8$ bit multiplexer, asserts two numbers on it's inputs, switches between them and checks the value of the output using asserts. The stimulus portion of the testbench is listed in Listing \ref{lst:test-mux}, and the resultant waveform is presented in Figure \ref{fig:test-mux}.

Inspection of the waveform demonstrates that the output is equal to \texttt{a}  when \texttt{sel} is low, and equal to \texttt{b} when \texttt{sel} is high. This is the expected behaviour, and the asserts in the testbench pass.


\lstinputlisting[language=verilog, caption={\texttt{test\_multiplexer.sv} Stimulus}, label={lst:test-mux},firstnumber=10,linerange={10-24}]{../hdl/sim/test_multiplexer.sv}

\begin{figure}[ht]
	\centering
	\begin{tikztimingtable} [xscale=2.0]
	a[7:0] & 3D{0x22} \\
	b[7:0] & 3D{0x5f} \\
	sel & L H L\\
	out[7:0] & 1D{0x22} 1D{0x5f}  1D{0x22} \\
\end{tikztimingtable}

	\caption{\texttt{test\_multiplexer.sv} Output}
	\label{fig:test-mux}
\end{figure}

\subsection{Register Design}
\subsubsection{Layout}

The register block is the most complex block of the design. At its heart it uses a dual port \gls{ram} block (with one dedicated read port, and one read/write port) to access data. Use of this configuration allows both registers to be read in a single clock cycle, as shown by Figure \ref{fig:proc-timing-diagram}. It should be noted however that the \gls{ram} inside the Cyclone IV \gls{fpga} does not support reading of new data during write, and so this is why there is a dedicated write cycle in the processors execution. Removal of this would result in incorrect data being read if an instruction requested the data from a register that was written to in the previous clock cycle.

The memory map of the registers is shown in Table \ref{tbl:reg-map}. Registers R1--R4 are general purpose computation registers. These are entirely application specific registers, and both reading and writing is legal for any of them. The contents of R4, however, does also map to the \glspl{led} on the \glspl{fpga} development board.

The U register stands for unity. This register is guaranteed to hold the constant necessary for an immediate to be loaded directly into a register using the \texttt{MULTI} command. This is necessary because by default immediates are treated as being a fractional constant, as discussed in Section \ref{sec:alu-design}. It is forbidden for any program to write to this value, as doing so will break the \texttt{LDI} (load immediate) command. The value of this register is initialised by the bitstream only, and resetting the processor using the reset switch will not reset its value. It should be noted that writing to this register is not prevented in hardware, but it is forbidden for any program to do so.

The Z register stands for zero. The value of this register is kept at zero, and writing to it should be done with extreme caution. Many of the higher level assembly commands internally rely on this register being zero when they are replaced with calls to \texttt{SUBLEQ}, and \texttt{MULTI}. Writing to Z, however, is not forbidden entirely as many of these higher level commands use it as a general purpose computation register, but they all guarantee to clear Z back to zero before exiting. Z is not initialised to zero by the bitstream, but instead the first command of the program must be \texttt{SUBLEQ Z Z} in order to clear it. This approach is taken so that the processor still functions correctly if it is reset using the reset switch whilst Z is non-zero.

The SW07 and SW8 registers are different from the others in that they do not map to internal storage in the \gls{fpga}. When the program attempts to read their value the value of switches 0--7, or switch 8 is returned instead, this is achieved by multiplexing the data outputs of the register bank. There do exist, however, registers inside the register memory at the addresses of the switch registers, this is because writing to the SW07 and SW8 registers is legal, but has no effect, physical registers need to exist in order to avoid an out of range write.


\begin{table}[ht]
	\centering
	\caption{Register map}
	\label{tbl:reg-map}
	\begin{tabular}{r | l}
		\toprule
		Address & Mnemonic \\
		\midrule
		0x0	& R1 \\
		0x1	& R2 \\
		0x2	& R3 \\
		0x3	& R4 / LED \\
		0x4	& U \\
		0x5	& Z \\
		0x6	& SW07 \\
		0x7	& SW8 \\
		\bottomrule
	\end{tabular}
\end{table}

\subsubsection{Implementation}
The registers are implemented as shown in Figure \ref{fig:reg-mem-arch}. The main block is the register memory block, this is a block of true dual port \gls{ram}. Data 1 is set to the value of the data stored at address 1 on each rising clock edge, and data 2 is set to the value of the data stored at address 2 if the write enable is false, otherwise the value stored at address 2 is replaced with the write data.

The \gls{led} register allows mirrors register R4 in the main memory bank, but allows the \glspl{led} to be constantly driven.

The switch multiplexers are three way multiplexers that multiplex between the register data, switches[0:7], and switches[8], dependant on the register address selected. The control signals for these multiplexers are derived from two register addresses. Internally the multiplexers are formed of two cascaded two input multiplexers, one to choose between the two registers of switches, and one to select between the switch values and the data register.

\begin{figure}[ht]
	\centering
	\def \nodeDist{2.5cm}
	\tikzstyle{normalBlock} = [rectangle, rounded corners, minimum width=\nodeDist, minimum height=\nodeDist, text centred, draw=black, fill=black!40]
	\tikzstyle{arrowNml} = [thick,->,>=stealth]
	\tikzstyle{arrowRev} = [thick,<-,>=stealth]
	\tikzstyle{arrowDbl} = [thick,<->,>=stealth]
	\tikzstyle{lineNml} = [thick]
	\begin{tikzpicture}[node distance=\nodeDist, scale=0.7, every node/.style={transform shape}]
	
	\newcommand{\buswidth}[5]{\draw ([xshift=#4,yshift=#5] #1) node [#3=.6ex,] {#2} +(45:-.8ex) -- +(45:.8ex)};
	
	
	% program memory
	\node (mem) [normalBlock, minimum height=7cm] at (0,0) {\shortstack{Register\\Memory}};
	\coordinate (addr1) at ([yshift=-1cm] mem.north west);
	\coordinate (addr2) at ([yshift=-2cm] mem.north west);
	
	\coordinate (d) at ([yshift=1cm] mem.south west);
	\coordinate (we) at ([yshift=2cm] mem.south west);
	\draw [arrowRev] (addr1) to node[above] {Address 1} ++(-1.5*\nodeDist,0) coordinate(addr1End);
	\draw [arrowRev] (addr2) to node[above] {Address 2} ++(-1.5*\nodeDist,0) coordinate(addr2End);
	\draw [arrowRev] (d) to node[above] {Write Data} ++(-1.5*\nodeDist,0) coordinate(dEnd);
	\draw [arrowRev] (we) to node[above] {Write Enable} ++(-1.5*\nodeDist,0) coordinate(weEnd);
	\buswidth{addr1End}{$3$}{below}{0.5cm}{0};
	\buswidth{addr2End}{$3$}{below}{0.5cm}{0};
	\buswidth{dEnd}{$8$}{below}{0.5cm}{0};
	\buswidth{weEnd}{$1$}{below}{0.5cm}{0};
	
	%LED register
	\node (ledReg) [normalBlock, minimum height=3cm, below= 0.5*\nodeDist of mem] {\shortstack{LED\\Register}};
	\draw [arrowNml] (dEnd) -- ++(0.75*\nodeDist,0) -- ++(0, -0.5*\nodeDist - 2cm) -- node[above] {Data} ++(0.75*\nodeDist,0);
	\coordinate (ledEnable) at ([yshift=1cm] ledReg.south west);
	\draw [arrowRev] (ledEnable) to node[above] {Enable} ++(-0.75*\nodeDist,0)  coordinate(ledEnableEnd);
	\node[left] at (ledEnableEnd) {\shortstack{(Write Enable) AND\\ (Address 2 == LED)}};
	
	\draw [arrowNml] (ledReg.east) -- node[above] {LEDs} ++(1.5*\nodeDist,0);
	\buswidth{ledReg.east}{$8$}{below}{0.5cm}{0};
	
	%Switch multiplexer 1
	\node (sw1Mux) [normalBlock, minimum height=3cm, right= of mem.north east, anchor = north west] {\shortstack{Switch\\Multiplexer}};
	
	\coordinate (switches) at ([yshift=-1cm] sw1Mux.north west);
	\coordinate (regdata) at ([yshift=-2cm] sw1Mux.north west);
	\draw [arrowRev] (switches) to node[above] {Switches} ++(-0.75*\nodeDist,0) coordinate(switchesEnd);
	\buswidth{switchesEnd}{$9$}{below}{0.5cm}{0};
	\draw [arrowRev] (regdata) to node[above] {Data 1} ++(-\nodeDist,0) coordinate(dataEnd);
	\buswidth{dataEnd}{$8$}{below}{0.5cm}{0};
	\draw [arrowNml] (sw1Mux.east) to node[above] {\shortstack{Register 1\\Data}} ++(\nodeDist,0);
	\buswidth{sw1Mux.east}{$8$}{below}{0.5cm}{0};
	
	\draw [arrowRev] (sw1Mux.north) -- ++(0,0.25*\nodeDist) coordinate(addrEnd);
	\node[above] (addrLabel) at (addrEnd) {Address 1};
	\buswidth{addrEnd}{$3$}{right}{0}{-0.25cm};
	
	%Switch multiplexer 2
	\node (sw2Mux) [normalBlock, minimum height=3cm, right= of mem.south east, anchor = south west] {\shortstack{Switch\\Multiplexer}};
	
	\coordinate (switches) at ([yshift=-1cm] sw2Mux.north west);
	\coordinate (regdata) at ([yshift=-2cm] sw2Mux.north west);
	\draw [arrowRev] (switches) to node[above] {Switches} ++(-0.75*\nodeDist,0) coordinate(switchesEnd);
	\buswidth{switchesEnd}{$9$}{below}{0.5cm}{0};
	\draw [arrowRev] (regdata) to node[above] {Data 1} ++(-\nodeDist,0) coordinate(dataEnd);
	\buswidth{dataEnd}{$8$}{below}{0.5cm}{0};
	\draw [arrowNml] (sw2Mux.east) to node[above] {\shortstack{Register 2\\Data}} ++(\nodeDist,0);
	\buswidth{sw2Mux.east}{$8$}{below}{0.5cm}{0};
	
	
	\draw [arrowRev] (sw2Mux.south) -- ++(0,-0.25*\nodeDist) coordinate(addrEnd);
	\node[below] (addrLabel) at (addrEnd) {Address 2};
	\buswidth{addrEnd}{$3$}{right}{0}{0.25cm};
	
	\end{tikzpicture}
	\caption{Register memory architecture}
	\label{fig:reg-mem-arch}
\end{figure}

\subsubsection{Testing}
In order to test the registers. I wrote a testbench which instantiates the register block, and presents its inputs with the signals which would be present if if were part of the overall system. The stimulus portion of the testbench is listed in Listing \ref{lst:test-regs}, and the resultant waveform is presented in Figure \ref{fig:test-regs}. Inspection of the stimulus portion shows that the registers are initially loaded with random values for testing, this allows the registers to be tested more rigorously since each register will have a unique value, rather than the initialisation value. A SystemVerilog task was written to perform one read/write cycle, and this uses asserts to test the output, and internal value of the register memory at each state.

For brevity, Figure \ref{fig:test-regs} shows a single iteration of the \texttt{rwTest} task, and is the result of executing the algorithm up to the commented out \texttt{stop} command on line 96. Inspection of the Figure shows that the values of registers at addresses $0x0$, and $0x1$ is requested, these are $0x72$, and $0xb2$ respectively. These values are correctly asserted on the output after one clock cycle. In addition, the value of the write data, $0x14$, is written to register $0x2$ on the third clock cycle as expected.

\lstinputlisting[language=verilog, caption={\texttt{test\_regs.sv} Stimulus}, label={lst:test-regs},firstnumber=82,linerange={82-107}]{../hdl/sim/test_regs.sv}

\begin{figure}[ht]
	\centering
	\begin{tikztimingtable} [xscale=2.0]
	clk & L L 4{C} \\
	cycle[1:0] & 2D{0x0} 2D{0x1} 2D{0x2} \\
	switches[8:0] & 1U 5D{0xdf} \\
	reg1\_addr[2:0] & 1U 5D{0x0} \\
	reg2\_addr[2:0] & 1U 5D{0x1} \\
	wr\_data[7:0] & 1U 5D{0x14} \\
	we & 2D{0x0} 2D{0x1} 2D{0x0} \\
	reg\_1[7:0] & 2U 4D{0x72} \\
	reg\_2[7:0] & 2U 4D{0xb2} \\
	leds[7:0] & 6D{0x0} \\
	mem[7][7:0] & 6D{0xb6} \\
	mem[6][7:0] & 6D{0xb5} \\
	mem[5][7:0] & 6D{0x0} \\
	mem[4][7:0] & 6D{0x8} \\
	mem[3][7:0] & 6D{0x30} \\
	mem[2][7:0] & 6D{0x10} \\
	mem[1][7:0] & 4D{0xb2} 2D{0x14} \\
	mem[0][7:0] & 6D{0x72} \\
\end{tikztimingtable}

	\caption{\texttt{test\_regs.sv} Output}
	\label{fig:test-regs}
\end{figure}

\section{\glsentryshort{alu} Design} \label{sec:alu-design}

The \gls{alu} design is relatively simple owing to the fact that the processor only contains two instructions, as discussed in Section \ref{sec:inst-format}. This means that the \gls{alu} only contains the implementation of the two instructions, and a multiplexer to assert the correct output based upon the opcode. This is illustrated graphically in Figure \ref{fig:alu-arch}. The multiplexer implementation uses a multiplier as discussed in Section \ref{sec:mux-design}

\texttt{SUBLEQ} is implemented using logic elements to form a subtracter, then the output of the subtracter is tested for the branch condition using a multiplier. The condition to branch is if the result is less than or equal to zero, we can test this using multiplication by $-1$. $-1 \times 0 = 0$, and $-1 \times [\text{Negative number}] = [\text{Positive number}]$, however $-1 \times [\text{Positive number}] = [\text{Negative number}]$, therefore the branch condition is the logical negation of the \gls{msb} of the multiplier output. This saves the use of several logic elements.

\texttt{MULTI} requires minimal overhead (only a single hardware multiplier -- a cost of 1 in the cost function), and provides loading of immediates without a multiplexer on the \gls{alu} input, as well as multiplication without a large loop of \texttt{SUBLEQ} instructions. Its implementation is straightforward, simply inferring a single hardware multiplier. The only slight complication is setting the significance of the input words. Whilst the data stored in the register has normal significance $[-2^7: 2^0]$, the immediate has significance $[-2^1: 2^{-3}]$. In order to allow use of 5 bit immediates to represent the fractional constants natively. When non fractional constants are loaded, the register which they are multiplied with is set to contain 4 in order to cancel out the effect.

\begin{figure}[ht]
	\centering
	\def \nodeDist{2.5cm}
	\tikzstyle{normalBlock} = [rectangle, rounded corners, minimum width=\nodeDist, minimum height=\nodeDist, text centred, draw=black, fill=black!40]
	\tikzstyle{arrowNml} = [thick,->,>=stealth]
	\tikzstyle{arrowRev} = [thick,<-,>=stealth]
	\tikzstyle{arrowDbl} = [thick,<->,>=stealth]
	\tikzstyle{lineNml} = [thick]
	\begin{tikzpicture}[node distance=\nodeDist, scale=0.7, every node/.style={transform shape}]
	
	\newcommand{\buswidth}[5]{\draw ([xshift=#4,yshift=#5] #1) node [#3=.6ex,] {#2} +(45:-.8ex) -- +(45:.8ex)};
	
	
	% MULTI
	\node (multi) [normalBlock, minimum height=3cm] at (0,0) {\shortstack{MULTI}};
	\coordinate (imm) at ([yshift=-1cm] multi.north west);
	\coordinate (reg1) at ([yshift=-2cm] multi.north west);
	\coordinate (multiResult) at ([yshift=-1cm] multi.north east);
	\coordinate (multiBranch) at ([xshift =0.5*\nodeDist, yshift=-2cm] multi.north east);
	\coordinate (multiMeet)  at ([xshift =\nodeDist, yshift=-1.5cm] multi.north east);
	
	\draw [arrowRev] (imm) to node[above] {Immediate} ++(-1.5*\nodeDist,0) coordinate(immEnd);
	\draw [arrowRev] (reg1) to node[above] {Register 1} ++(-1.5*\nodeDist,0) coordinate(reg1End);
	\buswidth{immEnd}{$8$}{below}{0.5cm}{0};
	\buswidth{reg1End}{$8$}{below}{0.5cm}{0};
	
	\draw [arrowNml] (multiResult) to node[above] {Result} ++(\nodeDist,0) -- (multiMeet);
	\draw [arrowNml] (multiBranch) to node[above] {`$0$'} ++(0.5*\nodeDist,0) -- (multiMeet);
	\buswidth{multiResult}{$8$}{below}{0.5cm}{0};
	\buswidth{multiBranch}{$1$}{below}{0.5cm}{0};
	
	% SUBLEQ
	\node (subleq) [normalBlock, minimum height=3cm, below = 1cm of multi] {\shortstack{SUBLEQ}};
	\coordinate (subleqReg1) at ([yshift=-1cm] subleq.north west);
	\coordinate (reg2) at ([yshift=-2cm] subleq.north west);
	\coordinate (subleqResult) at ([yshift=-1cm] subleq.north east);
	\coordinate (subleqBranch) at ([yshift=-2cm] subleq.north east);
	\coordinate (subleqMeet)  at ([xshift =\nodeDist, yshift=-1.5cm] subleq.north east);
	
	\draw [arrowRev] (subleqReg1) to node[above] {Register 1} ++(-0.75*\nodeDist,0) coordinate(subleqReg1Mid) |- (reg1);
	\draw [arrowRev] (reg2) to node[above] {Register 2} ++(-1.5*\nodeDist,0) coordinate(reg2End);
	\buswidth{reg2End}{$8$}{below}{0.5cm}{0};
	
	\draw [arrowNml] (subleqResult) to node[above] {Result} ++(\nodeDist,0) -- (subleqMeet);
	\draw [arrowNml] (subleqBranch) to node[above] {Branch} ++(\nodeDist,0) -- (subleqMeet);
	\buswidth{subleqResult}{$8$}{below}{0.5cm}{0};
	\buswidth{subleqBranch}{$1$}{below}{0.5cm}{0};
	
	%Mux
	\node (mux) [normalBlock, minimum height=7cm, right = 1.5*\nodeDist of multi.north east, anchor= north west] {\shortstack{Multiplexer}};
	\coordinate (in2) at ([yshift=1.5cm] mux.south west);
	\coordinate (in1) at ([yshift=-1.5cm] mux.north west);
	\coordinate (out) at (mux.east);
	\coordinate (sel) at (mux.south);
	
	\draw [arrowNml] (subleqMeet) -- (in2);
	\draw [arrowNml] (multiMeet) -- (in1);
	\buswidth{subleqMeet}{$9$}{below}{0.5cm}{0};
	\buswidth{multiMeet}{$9$}{below}{0.5cm}{0};
	
	\draw[arrowNml] (out) -- ++(0.5*\nodeDist,0) coordinate(outMeet);
	\draw[arrowNml] (outMeet) -- ++(0,0.5cm) coordinate(outResult) to node[above] {Result} ++(\nodeDist,0);
	\draw[arrowNml] (outMeet) -- ++(0,0.5cm) coordinate(outResult) to node[above] {Result} ++(\nodeDist,0);
	\draw[arrowNml] (outMeet) -- ++(0,-0.5cm) coordinate(outBranch) to node[above] {Branch} ++(\nodeDist,0);
	
	\buswidth{out}{$9$}{below}{0.5cm}{0};
	\buswidth{outResult}{$8$}{below}{0.5cm}{0};
	\buswidth{outBranch}{$1$}{below}{0.5cm}{0};
	
	\draw [arrowRev] (sel) -- ++(0,-0.25*\nodeDist) coordinate(selEnd);
	\node[below] (selLabel) at (selEnd) {Opcode};
	\buswidth{selEnd}{$1$}{right}{0}{0.25cm};
	
	\end{tikzpicture}
	\caption{\glsentryshort{alu} architecture}
	\label{fig:alu-arch}
\end{figure}


\subsubsection{Testing}
Due to the simplicity of the \gls{alu}, I chose to only test the \texttt{MULTI} operation separately and due to its simplicity test the \texttt{SUBLEQ} module in system level testing. I wrote a testbench which instantiates \texttt{MULTI}, presents stimulus data and asserts that the result produced is equal to manually calculated results. The stimulus portion of the testbench is listed in Listing \ref{lst:test-multi}, and the resultant waveform is presented in Figure \ref{fig:test-multi}. Inspection of the stimulus shows that for each case the multiplier correctly calculates the value of the of \texttt{result}, such that it is equal to the manually calculated value in each case. This is confirmed by the assertions.

\lstinputlisting[language=verilog, caption={\texttt{test\_multi.sv} Stimulus}, label={lst:test-multi},firstnumber=17,linerange={17-37}]{../hdl/sim/test_multi.sv}

\begin{figure}[ht]
	\centering
	\begin{tikztimingtable} [xscale=2.0]
	register[7:0] & 1D{0x6} 1D{0x8} 1D{0x80} \\
	immediate[4:0] & 1D{0x6} 1D{0xc} 1D{0x4} \\
	result[7:0] & 1D{0x4} 1D{0xc} 1D{0xc0} \\
\end{tikztimingtable}

	\caption{\texttt{test\_multi.sv} Output}
	\label{fig:test-multi}
\end{figure}

\subsection{System Level testing} \label{sec:sys-level-testing}.

To test the overall functionality of the system, a testbench was created to simulate switch input, and confirm the output result against a model of the affine transform calculated by the testbench. Using this method, the system was able to be tested for all $2^{16} = 65536$ possible inputs. Listing \ref{lst:test-picomips} shows the stimulus for the testing, which loops over all the possible input combinations. The stimulus in each case, along with expected values, is logged to a text file so that it can be used during the validation process as well. The key part of this stimulus is the SystemVerilog task \texttt{testAffineTransform}. This function calculates the expected value, mimics the switch input which the user would perform, and ensures that the outputs match using assertions.

Figure \ref{fig:test-picomips} shows the stimulus and output for one transform $x_1 = -1$ (0xff), $y_1 = 2$ (0x2). The expected outputs for this operation are $x_2 = 2$ (0x2), $y_2 = 13$ (0xd), so inspection of the \texttt{LED} signal in the waveform, along with \texttt{SW[7:0]}, show that the result is correct.

\lstinputlisting[language=verilog, caption={\texttt{test\_picoMIPS.sv} Stimulus}, label={lst:test-picomips},firstnumber=110,linerange={110-141}]{../hdl/sim/test_picoMIPS.sv}

\begin{figure}[ht]
	\centering
	\begin{tikztimingtable} [xscale=0.075]
	clk & 608{U} \\
	SW[9] & L L L L L L L L L H H H H H H H H H H H H H H H H H H H H H H H H H H H H H H H H H H H H H H H H H H H H H H H H H H H H H H H H H H H H H H H H H H H H H H H H H H H H H H H H H H H H H H H H H H H H H H H H H H H H H H H H H H H H H H H H H H H H H H H H H H H H H H H H H H H H H H H H H H H H H H H H H H H H H H H H H H H H H H H H H H H H H H H H H H H H H H H H H H H H H H H H H H H H H H H H H H H H H H H H H H H H H H H H H H H H H H H H H H H H H H H H H H H H H H H H H H H H H H H H H H H H H H H H H H H H H H H H H H H H H H H H H H H H H H H H H H H H H H H H H H H H H H H H H H H H H H H H H H H H H H H H H H H H H H H H H H H H H H H H H H H H H H H H H H H H H H H H H H H H H H H H H H H H H H H H H H H H H H H H H H H H H H H H H H H H H H H H H H H H H H H H H H H H H H H H H H H H H H H H H H H H H H H H H H H H H H H H H H H H H H H H H H H H H H H H H H H H H H H H H H H H H H H H H H H H H H H H H H H H H H H H H H H H H H H H H H H H H H H H H H H H H H H H H H H H H H H H H H H H H H H H H H H H H H H H H H H H H H H H H H H H H H H H H H H H H H H H H H H H H H H H H H H H H H H H H H H H H H H H H H H H H H H H H H H H H H H H H H H H H H H H H H H H H H H H H H H H H H H \\
	SW[8] & L L L L L L L L L H H H H H H H H H H H H H H H H H H H H H H H H H H H H H H H H H H H H H H H H H H H H H H H H H H H H H H H H H H H H H H H H H H H H H H H H H H H H H H H H H H H H H H H H H H H H H H H H H H H H L L L L L L L L L L L L L L L L L L L L L L L L L L L L L L L L L L L L L L L L L L L L L L L L L L L L L L L L L L L L L L L L L L L L L L L L L L L L L L L L L L L L L L L L L L L L L L L L L L L L H H H H H H H H H H H H H H H H H H H H H H H H H H H H H H H H H H H H H H H H H H H H H H H H H H H H H H H H H H H H H H H H H H H H H H H H H H H H H H H H H H H H H H H H H H H H H H H H H H H H L L L L L L L L L L L L L L L L L L L L L L L L L L L L L L L L L L L L L L L L L L L L L L L L L L L L L L L L L L L L L L L L L L L L L L L L L L L L L L L L L L L L L L L L L L L L L L L L L L L L H H H H H H H H H H H H H H H H H H H H H H H H H H H H H H H H H H H H H H H H H H H H H H H H H H H H H H H H H H H H H H H H H H H H H H H H H H H H H H H H H H H H H H H H H H H H H H H H H H H H L L L L L L L L L L L L L L L L L L L L L L L L L L L L L L L L L L L L L L L L L L L L L L L L L L L L L L L L L L L L L L L L L L L L L L L L L L L L L L L L L L L L L L L L L L L L L L L L L L L \\
	SW[7:0] & 9D{} 200D{0xff} 399D{0x2} \\
	LED[7:0] & 321D{0x0} 12D{} 24D{} 66D{0x2} 36D{} 149D{0xd} \\
\end{tikztimingtable}

	\caption{\texttt{test\_picoMIPS.sv} Output}
	\label{fig:test-picomips}
\end{figure}

\section{Altera DE2-115 Implementation and Validation} \label{sec:implementation}
Due to the extensive system level verification discussed in Section \ref{sec:sys-level-testing}, few problems were encountered during implementation. Upon testing the initial programming file, it was noted that the system did not respond to any input. Investigation of the setting in Quartus revealed that the pin mapping \texttt{.qsf} file had not been loaded properly and so the pin assignments were not correct. Upon fixing this issue, the design worked as expected.

The validation methodology was:
\begin{enumerate}
	 \item Select an $x$ and $y$ co-ordinate to test.
	 \item Look up the expected result in the output log produced by the system level testbench.
	 \item Input the data to the system under test.
	 \item Check the result produced.
\end{enumerate}
In particular, attention was paid to ensuring that a range of random values, as well as edge cases (co-ordinates close to $0$ and the limits of the integer size) were validated. In all cases the tested value matched the expected result.


\subsection{Decimal decoder}
In order to improve the ease of validation, a small SystemVerilog module was written to decode the signed 8-bit values on the switches and \glspl{led}, and display them in decimal on four seven segment displays. This module does not contribute to the cost figure of the design as it is instantiated outside of the picoMIPS module. The design of the module will not be covered in detail, as it is largely outside of the scope of the project.

\subsubsection{Testing}
Testing of the decimal decoder is shown in Listing \ref{lst:test-bin-bcd} and Figure \ref{fig:test-bin-bcd}. In the testbench stimulus, it can be seen that all $2^8 = 256$ possible integers are tested, however the shown waveform presented in this report, only covers the first two $-128$ and $-127$. Inspection of the waveform confirms that for $-128$ (0x80), the sign signal is true, and the tens, hundreds and units are correctly decoded to $1$, $2$, and $8$ respectively. The same is true for the second test, $-127$ (0x81).

\lstinputlisting[language=verilog, caption={\texttt{test\_bin\_to\_bcd.sv} Stimulus}, label={lst:test-bin-bcd},firstnumber=10,linerange={10-18}]{../hdl/sim/test_bin_to_bcd.sv}

\begin{figure}[ht]
	\centering
	\begin{tikztimingtable} [xscale=2.0]
	in[7:0] & 1D{0x80} 1D{0x81} \\
	sign & H H \\
	hundreds[3:0] & 2D{0x1} \\
	tens[3:0] & 2D{0x2} \\
	units[3:0] & 1D{0x8} 1D{0x7} \\
	disp[3][6:0] & 2D{0x40} \\
	disp[2][6:0] & 2D{0x6} \\
	disp[1][6:0] & 2D{0x5b} \\
	disp[0][6:0] & 1D{0x7f} 1D{0x7} \\
\end{tikztimingtable}

	\caption{\texttt{test\_bin\_to\_bcd.sv} Output}
	\label{fig:test-bin-bcd}
\end{figure}

\section{Conclusion}
All of the objectives noted in Section \ref{sec:intro} have been achieved. Objectives \ref{obj:design} and \ref{obj:imp} are realised through the design, and subsequent SystemVerilog implementation of the processor discussed in Section \ref{sec:design-verif}. Objectives \ref{obj:verif} and \ref{obj:valid} have been realised through the verification and validation of the design discussed in Sections \ref{sec:sys-level-testing} and \ref{sec:implementation} respectively.

Objective \ref{obj:minimal}, minimalisation of the design has been at the heart of the design philosophy throughout the project, and this culminates in the tiny cost Figure calculated by Equation \ref{eq:cost-complete}.

\begin{align}
\text{Cost} &= [\text{No. of Logic Elements}] + \max([\text{No. of 9-bit Multipliers used}] -2, 0) + \frac{ [\text{kBits of RAM}]}{1024} \times 30 \nonumber \\
&= 13 +  \max(13 -2, 0) + \frac{607}{1024} \times 30  \nonumber \\
&= 13 + 11 + 17.78  \nonumber \\
&= 41.78
\label{eq:cost-complete}
\end{align}

The created processor performs well and can easily be tailored to new applications due to the parametrised and modular design. The fact that the processor is accompanied by a powerful assembler aids in it's applicability to new applications.

In conclusion there are few ways in which the design could be improved without extending the scope. One key element that would make it more versatile would be adding support for a higher level programming language. This would allow the system to be tailored to new applications with ease. A C compiler would be the obvious choice, however this would encompass a large design effort, so porting a Forth runtime to the processor would likely be a more realistic goal.
